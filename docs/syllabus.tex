% Options for packages loaded elsewhere
\PassOptionsToPackage{unicode}{hyperref}
\PassOptionsToPackage{hyphens}{url}
%
\documentclass[
]{article}
\usepackage{lmodern}
\usepackage{amssymb,amsmath}
\usepackage{ifxetex,ifluatex}
\ifnum 0\ifxetex 1\fi\ifluatex 1\fi=0 % if pdftex
  \usepackage[T1]{fontenc}
  \usepackage[utf8]{inputenc}
  \usepackage{textcomp} % provide euro and other symbols
\else % if luatex or xetex
  \usepackage{unicode-math}
  \defaultfontfeatures{Scale=MatchLowercase}
  \defaultfontfeatures[\rmfamily]{Ligatures=TeX,Scale=1}
\fi
% Use upquote if available, for straight quotes in verbatim environments
\IfFileExists{upquote.sty}{\usepackage{upquote}}{}
\IfFileExists{microtype.sty}{% use microtype if available
  \usepackage[]{microtype}
  \UseMicrotypeSet[protrusion]{basicmath} % disable protrusion for tt fonts
}{}
\makeatletter
\@ifundefined{KOMAClassName}{% if non-KOMA class
  \IfFileExists{parskip.sty}{%
    \usepackage{parskip}
  }{% else
    \setlength{\parindent}{0pt}
    \setlength{\parskip}{6pt plus 2pt minus 1pt}}
}{% if KOMA class
  \KOMAoptions{parskip=half}}
\makeatother
\usepackage{xcolor}
\IfFileExists{xurl.sty}{\usepackage{xurl}}{} % add URL line breaks if available
\IfFileExists{bookmark.sty}{\usepackage{bookmark}}{\usepackage{hyperref}}
\hypersetup{
  hidelinks,
  pdfcreator={LaTeX via pandoc}}
\urlstyle{same} % disable monospaced font for URLs
\setlength{\emergencystretch}{3em} % prevent overfull lines
\providecommand{\tightlist}{%
  \setlength{\itemsep}{0pt}\setlength{\parskip}{0pt}}
\setcounter{secnumdepth}{-\maxdimen} % remove section numbering
\ifluatex
  \usepackage{selnolig}  % disable illegal ligatures
\fi

\author{}
\date{}

\begin{document}

\hypertarget{honors-seminar-in-machine-learning}{%
\section{Honors Seminar in Machine
Learning}\label{honors-seminar-in-machine-learning}}

Math 3094, Spring Semester 2021 University of Connecticut

\hypertarget{instructors}{%
\subsubsection{Instructors}\label{instructors}}

\begin{itemize}
\tightlist
\item
  \href{mailto:khlee@math.uconn.edu}{Kyu-Hwan Lee}
\item
  \href{mailto:jeremy.teitelbaum@uconn.edu}{Jeremy Teitelbaum}
\end{itemize}

\hypertarget{introduction}{%
\subsubsection{Introduction}\label{introduction}}

The interdisciplinary field known as Machine Learning or Data Science
draws together techniques from computer science, mathematics, and
statistics to extract meaning from data. In this course, we will discuss
some of the essential mathematical ideas in this field.

While our focus will be on the role of Calculus, Probability, and Linear
Algebra, we will introduce computational techniques using Python and the
Jupyter notebook environment, and some ideas from statistics, in order
to closely link theory and practice.

\hypertarget{schedule}{%
\subsubsection{Schedule}\label{schedule}}

The course will meet synchronously online on Tuesdays and Thursdays from
11:00 to 12:15 EST.

\hypertarget{topics}{%
\subsubsection{Topics}\label{topics}}

Topics will include Linear Regression, Gradient Descent, Logistic
Regression, Principal Component Analysis, and others as time permits.
The course will include both (online) lectures and lab sessions.

\hypertarget{assessment}{%
\subsubsection{Assessment}\label{assessment}}

Students will be expected to complete two projects, one due at midterm
time and one by the final. The final project may be a
continuation/extension of the midterm project. A typical project will be
an example data analysis written up using the Jupyter notebook. Projects
may be done individually or in groups of up to three people.

\hypertarget{resources}{%
\subsubsection{Resources}\label{resources}}

We will use the \href{http://campuswire.com}{Campus Wire} platform for
online help and discussions. Students enrolled in the course should
receive an electronic invite to the forum. Contact one of the professors
if you need access.

We will rely on the Python programming language, the Anaconda open
source data science platform, and the Jupyter notebook environment for
our computer work. All of this software can be obtained for Linux, Mac,
or Windows from the Anaconda website:
\href{http://www.anaconda.com}{www.anaconda.com}.

A very brief guide to installing the software is
\href{installing.md}{available here}.

There is no official textbook for the course. We will be providing notes
as we progress. The following texts may be useful as references.

\begin{itemize}
\item
  James, Witten, Hastie, Tibshirani. An Introduction to Statistical
  Learning (with Applications in R). This is an introductory text on
  machine learning with a more statistical emphasis than our course, and
  with computer examples in R instead of Python. It is an excellent and
  informative work, and it is \href{https://statlearning.com/}{available
  for free} from the book home page.
\item
  Bass, Alonso-Ruiz, Baudoin, et. al.\\
  \href{https://probability.oer.math.uconn.edu/3160-oer/}{UConn's Open
  Undergraduate Probability Text}. This is the (open source) textbook
  for UConn's undergraduate probability course, Math 3160.
\item
  Boyd, S. and Vandenberghe, L.
  \href{https://web.stanford.edu/~boyd/vmls/}{Introduction to Applied
  Linear Algebra}. This is a (free) introductory text on Linear Algebra
  with a focus on applications, especially to Least Squares.
\item
  Treil, S.
  \href{https://www.math.brown.edu/streil/papers/LADW/LADW.html}{Linear
  Algebra Done Wrong}. This is a more theoretical linear algebra text
  that treats important topics such as inner product spaces.
\item
  Bishop, C.
  \href{https://www.microsoft.com/en-us/research/people/cmbishop/prml-book/}{Pattern
  Recognition and Machine Learning} This is a (free) comprehensive look
  at machine learning; it claims to be aimed at ``advanced
  undergraduates or first year PhD students'' but is technically
  demanding.
\end{itemize}

\hypertarget{policy-statements}{%
\subsection{Policy Statements}\label{policy-statements}}

\hypertarget{academic-integrity}{%
\subsubsection{Academic Integrity}\label{academic-integrity}}

Students are bound by the university's policies on academic integrity.

\hypertarget{students-with-disabilities}{%
\subsubsection{Students with
disabilities}\label{students-with-disabilities}}

Students with disabilities should contact one of the instructors as soon
as possible to discuss any accommodations needed during the semester due
to a documented disabilities. If you have a documented disability for
which you wish to request academic accommodations and have not contacted
the Center for Students with Disabilities, please do so as soon as
possible. The CSD is located in Wilbur Cross, Room 204 and can be
reached at (860) 486-2020 or at csd@uconn.edu. Detailed information
regarding the process to request accommodations is available on the CSD
website at www.csd.uconn.edu.

\hypertarget{policy-against-discrimination-harassment-and-inappropriate-romantic-relationships}{%
\subsubsection{Policy Against Discrimination, Harassment and
Inappropriate Romantic
Relationships}\label{policy-against-discrimination-harassment-and-inappropriate-romantic-relationships}}

The University is committed to maintaining an environment free of
discrimination or discriminatory harassment directed toward any person
or group within its community -- students, employees, or visitors.
Academic and professional excellence can flourish only when each member
of our community is assured an atmosphere of mutual respect. All members
of the University community are responsible for the maintenance of an
academic and work environment in which people are free to learn and work
without fear of discrimination or discriminatory harassment. In
addition, inappropriate Romantic relationships can undermine the
University's mission when those in positions of authority abuse or
appear to abuse their authority. To that end, and in accordance with
federal and state law, the University prohibits discrimination and
discriminatory harassment, as well as inappropriate Romantic
relationships, and such behavior will be met with appropriate
disciplinary action, up to and including dismissal from the University.
More information is available at http://policy.uconn.edu/?p=2884.

\hypertarget{sexual-assault-reporting-policy}{%
\subsubsection{Sexual Assault Reporting
Policy}\label{sexual-assault-reporting-policy}}

To protect the campus community, all non-confidential University
employees (including faculty) are required to report assaults they
witness or are told about to the Office of Diversity \& Equity under the
Sexual Assault Response Policy. The University takes all reports with
the utmost seriousness. Please be aware that while the information you
provide will remain private, it will not be confidential and will be
shared with University officials who can help. More information is
available at http://sexualviolence.uconn.edu/.

\end{document}
