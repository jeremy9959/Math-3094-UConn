% Options for packages loaded elsewhere
\PassOptionsToPackage{unicode}{hyperref}
\PassOptionsToPackage{hyphens}{url}
\PassOptionsToPackage{dvipsnames,svgnames*,x11names*}{xcolor}
%
\documentclass[
]{article}
\usepackage{lmodern}
\usepackage{amssymb,amsmath}
\usepackage{ifxetex,ifluatex}
\ifnum 0\ifxetex 1\fi\ifluatex 1\fi=0 % if pdftex
  \usepackage[T1]{fontenc}
  \usepackage[utf8]{inputenc}
  \usepackage{textcomp} % provide euro and other symbols
\else % if luatex or xetex
  \usepackage{unicode-math}
  \defaultfontfeatures{Scale=MatchLowercase}
  \defaultfontfeatures[\rmfamily]{Ligatures=TeX,Scale=1}
\fi
% Use upquote if available, for straight quotes in verbatim environments
\IfFileExists{upquote.sty}{\usepackage{upquote}}{}
\IfFileExists{microtype.sty}{% use microtype if available
  \usepackage[]{microtype}
  \UseMicrotypeSet[protrusion]{basicmath} % disable protrusion for tt fonts
}{}
\makeatletter
\@ifundefined{KOMAClassName}{% if non-KOMA class
  \IfFileExists{parskip.sty}{%
    \usepackage{parskip}
  }{% else
    \setlength{\parindent}{0pt}
    \setlength{\parskip}{6pt plus 2pt minus 1pt}}
}{% if KOMA class
  \KOMAoptions{parskip=half}}
\makeatother
\usepackage{xcolor}
\IfFileExists{xurl.sty}{\usepackage{xurl}}{} % add URL line breaks if available
\IfFileExists{bookmark.sty}{\usepackage{bookmark}}{\usepackage{hyperref}}
\hypersetup{
  colorlinks=true,
  linkcolor=blue,
  filecolor=Maroon,
  citecolor=Blue,
  urlcolor=Blue,
  pdfcreator={LaTeX via pandoc}}
\urlstyle{same} % disable monospaced font for URLs
\usepackage{listings}
\newcommand{\passthrough}[1]{#1}
\lstset{defaultdialect=[5.3]Lua}
\lstset{defaultdialect=[x86masm]Assembler}
\usepackage{longtable,booktabs}
% Correct order of tables after \paragraph or \subparagraph
\usepackage{etoolbox}
\makeatletter
\patchcmd\longtable{\par}{\if@noskipsec\mbox{}\fi\par}{}{}
\makeatother
% Allow footnotes in longtable head/foot
\IfFileExists{footnotehyper.sty}{\usepackage{footnotehyper}}{\usepackage{footnote}}
\makesavenoteenv{longtable}
\setlength{\emergencystretch}{3em} % prevent overfull lines
\providecommand{\tightlist}{%
  \setlength{\itemsep}{0pt}\setlength{\parskip}{0pt}}
\setcounter{secnumdepth}{5}

%% pandoc-tablenos: required package
\usepackage{cleveref}
\ifluatex
  \usepackage{selnolig}  % disable illegal ligatures
\fi
\newlength{\cslhangindent}
\setlength{\cslhangindent}{1.5em}
\newlength{\csllabelwidth}
\setlength{\csllabelwidth}{3em}
\newenvironment{CSLReferences}[3] % #1 hanging-ident, #2 entry sp
 {% don't indent paragraphs
  \setlength{\parindent}{0pt}
  % turn on hanging indent if param 1 is 1
  \ifodd #1 \everypar{\setlength{\hangindent}{\cslhangindent}}\ignorespaces\fi
  % set line spacing
  % set entry spacing
  \ifnum #2 > 0
  \setlength{\parskip}{#3\baselineskip}
  \fi
 }%
 {}
\usepackage{calc} % for \widthof, \maxof
\newcommand{\CSLBlock}[1]{#1\hfill\break}
\newcommand{\CSLLeftMargin}[1]{\parbox[t]{\maxof{\widthof{#1}}{\csllabelwidth}}{#1}}
\newcommand{\CSLRightInline}[1]{\parbox[t]{\linewidth}{#1}}
\newcommand{\CSLIndent}[1]{\hspace{\cslhangindent}#1}

\author{}
\date{}

\begin{document}

\lstset{columns=fullflexible,breaklines=true,basicstyle=\small\ttfamily,backgroundcolor=\color{gray!10}}

\hypertarget{the-naive-bayes-classification-method}{%
\section{The Naive Bayes classification
method}\label{the-naive-bayes-classification-method}}

\hypertarget{introduction}{%
\subsection{Introduction}\label{introduction}}

In our discussion of Bayes Theorem, we looked at a situation in which we
had a population consisting of people infected with COVID-19 and people
not infected, and we had a test that we could apply to determine which
class an individual belonged to. Because our test was not 100 percent
reliable, a positive test result didn't guarantee that a person was
infected, and we used Bayes Theorem to evaluate how to interpret the
positive test result. More specifically, our information about the test
performance gave us the the conditional probabilities of positive and
negative test results given infection status -- so for example we were
given \(P(+|\mathrm{infected})\), the chance of getting a positive test
assuming the person is infected -- and we used Bayes Theorem to
determine \(P(\mathrm{infected}|+)\), the chance that a person was
infected given a positive test result.

The Naive Bayes classification method is a generalization of this idea.
We have data that belongs to one of two classes, and based on the
results of a series of tests, we wish to decide which class a particular
data point belongs to. For one example, we are given a collection of
product reviews from a website and we wish to classify those reviews as
either ``positive'' or ``negative.'' This type of problem is called
``sentiment analysis.'' For another, related example, we have a
collection of emails or text messages and we wish to label those that
are likely ``spam'' emails. In both of these examples, the ``test'' that
we will apply is to look for the appearance or absence of certain key
words that make the text more or less likely to belong to a certain
class. For example, we might find that a movie review that contains the
word ``great'' is more likely to be positive than negative, while a
review that contains the word ``boring'' is more likely to be negative.

The reason for the word ``naive'' in the name of this method is that we
will derive our probabilities from empirical data, rather than from any
deeper theory. For example, to find the probability that a negative
movie review contains the word ``boring,'' we will look at a bunch of
reviews that our experts have said are negative, and compute the
proportion of those that contain the word boring. Indeed, to develop our
family of tests, we will rely on a training set of already classified
data from which we can determine estimates of probabilities that we
need.

\hypertarget{an-example-dataset}{%
\subsection{An example dataset}\label{an-example-dataset}}

To illustrate the Naive Bayes algorithm, we will work with the
``Sentiment Labelled Sentences Data Set''
({[}\protect\hyperlink{ref-sentences}{1}{]}). This dataset contains 3
files, each containing 1000 documents labelled \(0\) or \(1\) for
``negative'' or ``positive'' sentiment. There are 500 of each type of
document in each file. One file contains reviews of products from
amazon.com; one contains yelp restaurant reviews, and one contains movie
reviews from imdb.com.

Let's focus on the amazon reviews data. Here are some samples:

\begin{lstlisting}
So there is no way for me to plug it in here in the US unless I go by a converter.  0
Good case, Excellent value. 1
Great for the jawbone.  1
Tied to charger for conversations lasting more than 45 minutes.MAJOR PROBLEMS!! 0
The mic is great.   1
I have to jiggle the plug to get it to line up right to get decent volume.  0
If you have several dozen or several hundred contacts, then imagine the fun of sending each of them one by one. 0
If you are Razr owner...you must have this! 1
Needless to say, I wasted my money. 0
What a waste of money and time!.    0
\end{lstlisting}

As you can see, each line consists of a product review followed by a
\(0\) or \(1\); in this file the review is separated from the text by a
tab character.

\hypertarget{bernoulli-tests}{%
\subsection{Bernoulli tests}\label{bernoulli-tests}}

We will describe the ``Bernoulli'' version of a Naive Bayes classifier
for this data. The building block of this method is a test based on a
single word. For example, let's consider the word \textbf{great} among
all of our amazon reviews. It turns out that \textbf{great} occurs \(5\)
times in negative reviews and \(92\) times in positive reviews among our
\(1000\) examples. So it seems that seeing the word \textbf{great} in a
review makes it more likely to be positive. The appearances of great are
summarized in \cref{tbl:great} . We write \textbf{\textasciitilde great}
for the case where \textbf{great} does \emph{not} appear.

\begin{longtable}[]{@{}llll@{}}
\caption{Ocurrences of \textbf{great} by type of review
.\label{tbl:great}}\tabularnewline
\toprule
& + & - & total\tabularnewline
\midrule
\endfirsthead
\toprule
& + & - & total\tabularnewline
\midrule
\endhead
\textbf{great} & 92 & 5 & 97\tabularnewline
\textasciitilde{}\textbf{great} & 408 & 495 & 903\tabularnewline
total & 500 & 500 & 1000\tabularnewline
\bottomrule
\end{longtable}

In this data, positive and negative reviews are equally likely so
\(P(+)=P(-)=.5\) From this table, and Bayes Theorem, we obtain the
empirical probabilities (or ``naive'' probabilities).

\[
P(\mathbf{great} | +) = \frac{92}{500} = .184
\]

and

\[
P(\mathbf{great}) = \frac{97}{1000} = .097
\]

Therefore

\[
P(+|\mathbf{great}) = \frac{.184}{.097}(.5) = 0.948.
\]

In other words, \emph{if} you see the word \textbf{great} in a review,
there's a 95\% chance that the review is positive.

What if you \emph{do not} see the word \textbf{great}? A similar
calculation from the table yields

\[
P(+|\mathbf{~great}) = \frac{408}{903} = .452
\]

In other words, \emph{not} seeing the word great gives a little evidence
that the review is negative (there's a 55\% chance it's negative) but
it's not that conclusive.

The word \textbf{waste} is associated with negative reviews. It's
statistics are summarized in \cref{tbl:waste}.

\begin{longtable}[]{@{}llll@{}}
\caption{Ocurrences of \textbf{waste} by type of review
.\label{tbl:waste}}\tabularnewline
\toprule
& + & - & total\tabularnewline
\midrule
\endfirsthead
\toprule
& + & - & total\tabularnewline
\midrule
\endhead
\textbf{waste} & 0 & 14 & 14\tabularnewline
\textasciitilde{}\textbf{waste} & 500 & 486 & 986\tabularnewline
total & 500 & 500 & 1000\tabularnewline
\bottomrule
\end{longtable}

Based on this data, the ``naive'' probabilities we are interested in
are:

\begin{align*}
P(+|\mathbf{waste}) &= 0\\
P(+|\mathbf{~waste}) &= .51
\end{align*}

In other words, if you see \textbf{waste} you definitely have a negative
review, but if you don't, you're only slightly more likely to have a
positive one.

What about combining these two tests? Or using even more words? We could
analyze our data to count cases in which both \textbf{great} and
\textbf{waste} occur, in which only one occurs, or in which neither
occurs, within the two different categories of reviews, and then use
those counts to estimate empirical probabilities of the joint events.
But while this might be feasible with two words, if we want to use many
words, the number of combinations quickly becomes huge. So instead, we
make a basic, and probably false, assumption, but one that makes a
simple analysis possible.

\textbf{Assumption:} We assume that the presence or absence of the words
\textbf{great} and \textbf{waste} in a particular review (positive or
negative) are independent events. More generally, given a collection of
words \(w_1,\ldots, w_k\), we assume that their occurences in a given
review are independent events.

Independence means that we have \begin{align*}
P(\mathbf{great},\mathbf{waste}|\pm) &= P(\mathbf{great}|\pm)P(\mathbf{waste}|\pm)\\
P(\mathbf{great},\mathbf{~waste}|\pm) &= P(\mathbf{great}|\pm)P(\mathbf{~waste}|\pm)\\
 &\vdots \\
\end{align*}

To generalize this, suppose that we have extracted keywords
\(w_1,\ldots, w_k\) from our data and we have computed the empirical
values \(P(w_{i}|+)\) and \(P(w_{i}|-)\) by counting the fraction of
positive and negative reviews that contain the word \(w_{i}\):

\[
P(w_{i}|\pm) = \frac{\hbox{\rm number of $\pm$ reviews that mention $w_{i}$}}{\hbox{\rm number of $\pm$ reviews total}}
\]

Notice that we only count \emph{reviews}, not \emph{ocurrences}, so that
if a word occurs multiple times in a review it only contributes 1 to the
count. That's why this is called the \emph{Bernoulli} Naive Bayes -- we
are thinking of each keyword as yielding a yes/no test on each review.

Given a review, we look to see whether each of our \(k\) keywords
appears or does not. We encode this information as a vector of length
\(k\) containing \(0\)'s and \(1\)'s indicating the absence or presence
of the \(k\)th keyword. Let's call this vector the \emph{feature vector}
for the review.

For example, if our keywords are \(w_1=\mathbf{waste}\),
\(w_2=\mathbf{great}\), and \(w_3=\mathbf{useless}\), and our review
says

\begin{lstlisting}
This phone is useless, useless, useless!  What a waste!
\end{lstlisting}

then the associated feature vector is \(f=(1,0,1)\).

If a review has an associated feature vector \(f=(f_1,\ldots, f_k)\),
the probability of that feature vector ocurring within one of the
\(\pm\) classes is \[
P(f|\pm) = \prod_{i: f_{i}=1} P(w_{i}|\pm)\prod_{i: f_{i}=0}(1-P(w_{i}|\pm))
\]

\hypertarget{bibliography}{%
\section*{References}\label{bibliography}}
\addcontentsline{toc}{section}{References}

\hypertarget{refs}{}
\begin{CSLReferences}{0}{0}
\leavevmode\hypertarget{ref-sentences}{}%
\CSLLeftMargin{{[}1{]} }
\CSLRightInline{\textsc{U.C. Irvine ML Repository}. {Sentiment Labelled
Sentences Data Set}.Available at
\url{https://archive.ics.uci.edu/ml/datasets/Sentiment+Labelled+Sentences}.}

\end{CSLReferences}

\end{document}
